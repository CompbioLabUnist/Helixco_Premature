% !TeX spellcheck = en_US
% !TeX encoding = UTF-8
\documentclass[a4paper]{article}
\usepackage{graphics, graphicx}
\usepackage{fancyvrb, enumerate}
\usepackage{amsmath, amssymb, amscd, amsfonts}
\usepackage{geometry}
\usepackage{multirow}
\usepackage{url}
\usepackage{listings, listing}
\usepackage{color}
\usepackage{mathptmx}
\usepackage[numberedbib]{apacite}
\usepackage[style=iso]{datetime2}

\geometry
{
    top = 20mm,
    bottom = 20mm,
    left = 20mm,
    right = 20mm
}

\title{Microbiome Premature}
\author{Jaewoong Lee}
\date{\today}

\begin{document}
   	\maketitle
    \newpage

    \tableofcontents
    \listoftables
    \listoffigures
    \newpage

    \section{Introduction}
        \subsection{Microbiome}
            After the Human Genome Project was finished, the microorganisms which live along humans, as known as the microbiota, are considered overwhelmed human cells \cite{micro1}. Moreover, the microbiome, the collective genome from these microbiota \cite{micro2}, serve as the trait of individual have not to evolve on their own \cite{micro1}. Furthermore, human microbiome is effected by host's life style as figure \ref{fig:microbiome}.

            \begin{figure}[p]
                \centering
                \includegraphics[width=0.4 \linewidth]{figures/microbiome.jpg}
                \caption{Concept of a core human microbiome. \protect \cite{micro1}}
                \label{fig:microbiome}
            \end{figure}

        \subsection[rRNA]{Ribosomal RNA}
            Ribosomal RNA (rRNA) plays the main roles in a cell. This main roles include mRNA selection, tRNA binding, proof-reading, factor binding, and \textit{et cetera} \cite{rrna2}. Because of its momentous roles, rRNA could be preserved amongst whole bacteria throughout the evolution.

        \subsection{Premature}
            Premature (PTB; stands for Preterm Birth) is the birth of a baby earlier than 37 gestational weeks, as Figure \ref{fig:ptb} \cite{premature1}. Premature infants have more risk such as hearing problems and sight problems.

            \begin{figure}[p]
                \centering
                \includegraphics[width=0.4 \linewidth]{figures/premature.png}
                \caption{Definition of Premature \protect \cite{premature1}}
                \label{fig:ptb}
            \end{figure}

    \section{Materials}
        \subsection{16S rRNA Sequencing}
            rRNA has been kept among bacteria; thus, 16S rRNA exists in almost bacteria, and its functions has not changed over time. Also, 16S rRNA is large enough for bioinformatics \cite{16S2}. Hence, 16S rRNA sequencing is the reference method for bacterial taxonomy classification and identification \cite{16S1}.

            There are three databases which for machine learning: Helixco data, EBI data, and HMP data. Metadata of these databases is as table \ref{tb:metadata}.

            \begin{table}[p]
                \centering
                \caption{Metadata of Data}
                \label{tb:metadata}
                \begin{tabular}{c|ccc}
    Data & Participants & Samples & Remarks \\ \hline
    First & 24 & 107 & - \\
    Second & 35 & 288 & - \\
    Stool & 63 & 126 & Stool \\
    EBI & 18 & 1016 & Only Normal \\
    HMP & 1572 & 9205 & Only Premature \\
\end{tabular}

            \end{table}

            \subsubsection{Helixco Data}

            \subsubsection[EBI Data]{European Bioinformatics Institute Data}
                EBI data was collected by European Bioinformatics Institute (EBI) \cite{validate1}. EBI data aimed to compare Cesarean section birth and vaginal birth; thus, every participants in EBI data is on term, not PTB.

            \subsubsection[HMP Data]{Human Microbiome Project Data}
                HMP data was collected by Human Microbiome Project (HMP) \cite{validate2}. HMP data aimed to compare PTB and on-term birth; thus, every participants in HTMP data is PTB.
    \section{Methods}
        \subsection{QIIME 2}
            QIIME 2 is a next-generation microbiome bioinformatics platform which is extensible, free, open-source, and community developed \cite{qiime1, qiime2, qiime3}.

            \begin{figure}[p]
                \centering
                \includegraphics[width=0.6 \linewidth]{figures/qiime.png}
                \caption{Workflow of QIIME2}
                \label{fig:qiime2}
            \end{figure}

        \subsection{Denoising Algorithms}
            There are two denoising algorithms which are provided by QIIME as figure \ref{fig:denoising}: DADA2 \cite{DADA2} and Deblur \cite{Deblur1}.

            \begin{figure}[p]
                \centering
                \includegraphics[width=0.3 \linewidth]{figures/tikz/denoising.pdf}
                \caption{Denoising Algorithms}
                \label{fig:denoising}
            \end{figure}

            \subsubsection{DADA2}
                DADA2 is an open-source software package for modeling and correcting Illumina-sequenced amplicon errors \cite{DADA2}.

            \subsubsection{Deblur}
                Deblur is a software packages which uses error profiles to obtain putative error-free sequences from Illumina MiSeq and HiSeq sequencing platforms \cite{Deblur1}.

        \subsection{Taxonomy Classification Algorithms}
            There are two taxonomy classification algorithms which are provided by QIIME as figures \ref{fig:taxonomy}: Greengenes \cite{greengenes1} ans SILVA \cite{silva1, silva2}.

            \begin{figure}[p]
                \centering
                \includegraphics[width=0.3 \linewidth]{figures/tikz/taxonomy.pdf}
                \caption{Taxonomy Classification Algorithms}
                \label{fig:taxonomy}
            \end{figure}

            \subsubsection{Greengenes}
                Greengenes (GG) is a chimera-checked 16S rRNA gene database \cite{greengenes1}.

            \subsubsection{SILVA}
                SILVA is a comprehensive web resource for up-to-date, quality-controlled databases of aligned rRNA gene sequences from the Bacteria domains \cite{silva1, silva2}.

        \subsection{Mothur}
            Mothur is an open-source software package for bioinformatics data processing, especially for the analysis of DNA from microbes \cite{mothur1}.

        \subsection[t-SNE]{t-distributed Stochastic Neighbor Embedding}
            T-distributed stochastic neighbor embedding (t-SNE) visualizes high-dimensional data by giving each data-point a location in a two-dimensional map \cite{tsne1}.

            \begin{figure}[p]
                \centering
                \includegraphics[width=0.5 \linewidth]{figures/tsne.png}
                \caption{t-SNE Visualizations of handwritten digits from MNIST data \protect \cite{tsne1}}
                \label{fig:tsne}
            \end{figure}

        \subsection{Python Packages}
            \subsubsection{Pandas}
                Pandas is a Python library of rich data structure and tools for working with structured data sets \cite{pandas1}.

            \subsubsection{Scikit-Learn}
                Scikit-learn is a Python module which integrating a wide range of state-of-the-art machine learning algorithms for medium-scale supervised and unsupervised problems \cite{sklearn1}.

            \subsubsection{Matplotlib}
                Matplotlib is a two-dimensional graphics package used for Python for image generation \cite{matplotlib1}.

            \subsubsection{Seaborn}
                Seaborn is a Python data visualization library based on Matplotlib \cite{seaborn1}.

    \section{Results}
        \subsection{t-SNE for Brief Information}
            To compare three databases, workflow, which as figure \ref{fig:workflow-brief}, was executed.

            \begin{figure}[p]
                \centering
                \includegraphics[width=0.3 \linewidth]{figures/tikz/brief.pdf}
                \caption{Workflow of t-SNE for Brief Information}
                \label{fig:workflow-brief}
            \end{figure}

            \begin{figure}[p]
                \centering
                $\begin{array}{cc}
                    \includegraphics[width=0.3 \linewidth]{figures/tikz/brief/DADA2.gg.pdf}
                    &
                    \includegraphics[width=0.3 \linewidth]{figures/tikz/brief/DADA2.silva.pdf}
                    \\
                    \mbox{(a) DADA2 + GG} & \mbox{(b) DADA2 + SILVA} \\

                    \includegraphics[width=0.3 \linewidth]{figures/tikz/brief/Deblur.gg.pdf}
                    &
                    \includegraphics[width=0.3 \linewidth]{figures/tikz/brief/Deblur.silva.pdf}
                    \\
                    \mbox{(c) Deblur + GG} & \mbox{(d) Deblur + SILVA}
                \end{array}$
                \caption{Count of Intersected Taxa Information}
                \label{fig:inter-brief}
            \end{figure}

            \begin{figure}[p]
                \centering
                $\begin{array}{cc}
                    \includegraphics[width=0.3 \linewidth]{figures/TSNE/Brief/DB.DADA2.gg.png}
                    &
                    \includegraphics[width=0.3 \linewidth]{figures/TSNE/Brief/DB.DADA2.silva.png}
                    \\
                    \mbox{(a) DADA2 + GG} & \mbox{(b) DADA2 + SILVA} \\

                    \includegraphics[width=0.3 \linewidth]{figures/TSNE/Brief/DB.Deblur.gg.png}
                    &
                    \includegraphics[width=0.3 \linewidth]{figures/TSNE/Brief/DB.Deblur.silva.png}
                    \\
                    \mbox{(c) Deblur + GG} & \mbox{(d) Deblur + SILVA}
                \end{array}$
                \caption{t-SNE for Brief Information}
                \label{fig:tsne-brief}
            \end{figure}

    \section{Discussion}

    \bibliographystyle{apacite}
    \bibliography{reference}
\end{document}